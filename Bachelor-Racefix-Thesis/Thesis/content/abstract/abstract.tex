\begin{alwayssingle} 
\pagestyle{empty}
\begin{center}
\vspace*{1.5cm}
{\Large \bfseries  Abstract}
\end{center}
\vspace{0.5cm}

Software has become an integral part of our lives, it has pervaded almost every
domain essential to civilization. The implications of the axiomatic saying ``the
only constant is change'' which has been uttered and reattributed countless
times over the course of history is painfully obvious in software. The changes
required are not one-dimensional, the need for change can be driven by an update
of the requirements, the need for portability to a new platform, better
hardware. The latter reason has departed from its previous pattern of evolution.
Namely, CPU performance no longer improves through mere frequency increase, but
by introducing multiple cores. This fact leads to an interesting problem: we can
no longer rely on frequency scaling to improve our software. Thus we have be
able to make the changes to old software  that are necessary exploit these newly
found resources.

This thesis explores the concept of automated refactorings that aid the
programmer in retrofitting legacy software that no longer scales with the new
generations of microprocessors. To that end, it presents an automated tool that
fixes data-races encountered during loop parallelization.

\end{alwayssingle}