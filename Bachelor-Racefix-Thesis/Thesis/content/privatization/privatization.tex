\graphicspath{{content/privatization/figures/}}

\mychapter{Privatization}{chap:privatization}
\epigraph{\textit{I love fool's experiments. I am always making them!}}
{---\textsc{Charles Darwin}} 

In the introduction~\ref{chap:introduction} I have mentioned that we are
employing our \emph{ThreadPrivate} class instead of Java's standard
\emph{ThreadLocal} to handle data privatization. The motivation for this is
quite straightforward: during the manual parallelization, for me, as a human, it was
easy to keep track of the high level model of the program and realize that the
data that requires privatization does not have to be in any special state for
the parallel computation to execute correctly. Unfortunately, no present
automated tool can do this kind of high level analysis. Thus we had to come up
with a generic solution that can be easily applied by a machine. We save the
state of every object target to privatization prior to entering the parallel
loop context and offer a mechanism akin to \emph{Object.clone()} to freshly
recreate it for each thread.

\mysection{Library classes provided by us}{sec:ThreadPrivate-Privatizer}

\mysubsection{ThreadPrivate}{ssec:threadprivate}

\emph{ThreadPrivate} extends \emph{ThreadLocal} and offers a mechanism for
propagating the state of the encapsulated objects have to the context of the
threads. This is achieved by saving the reference to every object that is set
from the main thread with \emph{ThreadPrivate.set(value)}. The storing and
retrieval of the objects is the job of our data structure called
\emph{Privatizer}.

\emph{ThreadPrivate} overrides and makes the \emph{initialValue()} hook of
\emph{ThreadLocal} final, because it is within that method that the interaction
with \emph{Privatizer} takes place. Clients who wish to provide an initial value
will have to use the \emph{initValue()} method provided by \emph{ThreadPrivate}
which is perfectly analogous to the original \emph{initialValue()}. The source
code for \emph{ThreadPrivate} can be found in appendix~\ref{ap:TP}.

Unfortunately, ThreadPrivate can be used only to encapsulate objects of class
\emph{Privatizable} and an ever-growing base of standard library objects (i.e.
Integer, Collections, etc).

\mysubsection{Privatizable}{ssec:privatizable}
\emph{Privatizable} objects are used by \emph{Privatizer} to ensure state
preservation from the main thread and propagation towards any additional
threads. The source code for this interface can be found in
listing~\ref{code:privatizable}. The method \emph{createPrivate()} serves the
purpose of a factory method for itself. After its creation, the object is passed
onto the method \emph{populatePrivate} which is responsible with the
initialization of the new object's state, if necessary.

\begin{lstlisting}[caption={Privatizable source
code}, label = {code:privatizable}]
public interface Privatizable<T extends Privatizable<T>> {
	public T createPrivate();
	public void populatePrivate(T p);
}
\end{lstlisting}

\mysubsection{Privatizer}{ssec:privatizer}

Privatizer is responsible with the proper storage and retrieval of privatizable
objects. This is done by maintaining an \emph{IdentityHashMap} for each thread
that maps an object (the original object who's state we're saving) to its
corresponding copy, which is then used in thread. The source code for privatizer
can be found in appendix~\ref{ap:Privatizer}.

At this point in time \emph{Privatizer} is highly experimental and will be
subject to improvements.

\mysubsection{Example of usage}{ssec:example-threadprivate}
Using the above described classes the example presented in
listing~\ref{code:jmol:SticksRenderer-TL-DTO} can be rewritten to use
\emph{ThreadPrivate} as follows~\ref{code:privatizer:jmol-dto}:

\begin{lstlisting}[caption={Using a data transfer object}, label =
{code:privatizer:jmol-dto}]
public class SticksRenderer extends ShapeRenderer {
	public static class SRData implements  Privatizable<SRData>{
		private Atom atomA, atomB;
  		private Bond bond;
  		private int xA, yA, zA;
  		private int xB, yB, zB;
  		private int dx, dy;
  		private short colixA, colixB;
		private boolean isAntialiased;
  		private boolean slabbing;
  
  		private final Vector3f x = new Vector3f();
  		private final Vector3f y = new Vector3f();
  		private final Vector3f z = new Vector3f();
  		
		@Override
		public SRData createPrivate() {
			return new SRData();
		}

		@Override
		public void populatePrivate(SRData t) {
		//here we would have to assign initial values to the new object, 
		//if necessary
		}
	}
	
	ThreadLocal<SRData> data = new ThreadLocal<SRData>{
		@Override
		public SRData initialValue() {
			return new SRData();
		}
	};
}
\end{lstlisting}

\mysection{Computing privatization
information}{sec:computing-privatization-info}

This part of the algorithm uses the trace information gathered during the access
trace presented in~\ref{chap:access-trace} to compute the privatization scheme.
Basically it aims to encapsulate at the highest level possible (i.e. trying to
find the classes that encapsulate most data marked for privatization) excluding
the class that contains the computation and fields that are \lcds.

\mysection{Refactoring}{sec:refactoring}
The truly difficult part of any refactoring is the analysis, the checking of
preconditions which dominated the discourse topic of this thesis up until now.
Once you have this information building a refactoring on top of the
\emph{eclipse} platform becomes straightforward. The general overview how a
refactoring works in eclipse is described in the theory
section~\ref{th:eclipse-ref} and there is little use to giving more than an
abstract description of our refactorings.

\paragraph{The \emph{ThreadPrivate} refactoring} takes as input a map from
classes that are to implement the \emph{Privatizable} interface to a set of fields of that class that are to be
made \emph{ThreadPrivate}. The refactoring's responsibility is to change all
references to fields to go through \emph{ThreadPrivate's} getters and setters
and generate the implementation of the \emph{createPrivate()} and
\emph{populatePrivate(T)} methods. Last, notifying the user if the changes
generate any compilation errors.

\paragraph{The \emph{ThreadLocal} refactoring} takes as input the fully
qualified names of the fields to be converted; it updates all necessary references to use getters
and setters and it writes the \emph{initialValue()} method to return the value
assigned during initialization.

The two refactorings can easily work independently from our tool and have been
integrated in the UI of eclipse giving us the chance to test them manually. They
have been invoked on several fields from Jmol and performed the transformations
as expected.



