\mychapter{Theoretical constructs}{chap:theory}
\epigraph{\textit{These aren't the droids you're looking for.}}%
{---\textsc{Obi-Wan Kenobi}} 

\mysection{Refactoring}{th:refactoring}
In~\cite{Fowler-Refactoring} Martin Fowler defines refactoring as a
``disciplined technique for restructuring an existing body of code, altering its
internal structure without changing its external behavior''. It has pervaded the
community of agile software development because it is as an excellent way
of coping with change, both by making it easier to provide elegant solutions for
immediate problems and prepare the systems for future change.

\mysubsection{Automated refactorings}{th:auto-refacorings}
In the last years many refactorings have been automated as a means of freeing
the programmer of repetitive, error prone tasks. Modern IDEs like
\emph{eclipse}~\cite{eclipse-site} offer user friendly support for complex
refactorings like \emph{extract method, pull up, push down, extract interface}.
Also, researchers have been proposing new kinds of
refactorings~\cite{ReLooper,Immutator,Ref-Reentrancy,Ref-SchaferSDT11,Ref-conc-lib,Ref-NestedHeapRegions}
which try to tackle more complex problems than those readily available in our
IDEs.

\mysubsection{Eclipse refactoring user extensions}{th:eclipse-ref}
The \emph{eclipse IDE} is notorious for its plugin architecture and support for
user extensions, which does not shy away from user-defined refactorings. A
comprehensible tutorial for using the eclipse API can be found
at~\cite{eclipse-refactoring}.


\mysection{Concurrency and parallelism}{th:conc-and-parallelism}
Concurrency is defined as two or more tasks/actions run in overlapping time
periods. A concurrent application will have two or more threads in progress at
some time. This can mean that the application has two threads that are being
swapped in and out by the operating system on a single core processor. These
threads will be �in progress�---each in the midst of its execution---at the same
time.

Parallelism is defined as two or more tasks/actions run simultaneously.
Obviously, parallelism is a subset of concurrency~\cite{TAOC}.

\mysubsection{Data-race}{th:data-race}
Data-races arise in concurrent applications when two or more threads contend for
the same memory location at the same time; e.g. when two threads try to update
the same memory location at the same time; or when one thread updates and
another one tries to read.

\mysubsection{Thread confinement}{th:thread-confinement}
Data-races arise when threads try to access shared memory. A simple way to solve
these data races is by not sharing memory. If only one single thread can access
some data then no additional measures have to be taken to avoid
data-races~\cite{JCP-book}.

\mysubsection{Mutual exclusion}{th:locking}
In order to \emph{correctly} communicate in memory, threads have to protect
access to the shared memory locations. This is done by making sure that only one
thread at a time has access to said memory location i.e. \emph{exclusive}
access, or more often used \emph{synchronized} access. Synchronization has a
major drawback as the way it is usually implemented causes performance
bottlenecks and \emph{dead-locks} (when two thread wind up blocking each other's
access to shared memory due to programming errors). When programming to
eliminate data-races trade-off between memory usage/performance varies in
tandem with the use of thread confinement/synchronization.

\mysection{Java concurrency libraries}{th:java-conc-lib}
The Java programming language is shipped with an extensive library that supports
concurrent programming. Starting from thread creation and execution to thread
safe collections and synchronization mechanisms. Also available, is a library
called JSR166~\cite{JSR166-site} that is still in development; it offers
very good and convenient support for parallel execution. Currently, it is
scheduled to be included in the official release of Java 8.

\mysubsection{ParallelArray}{th:parallel-array}
\emph{ParallelArray}~\cite{PA-site} is a data structure included in the JSR166
library that gives the user an easy way of applying parallel operations, reduce
operations on the elements of an array. The framework assures thread scheduling
and execution while the programmer is responsible to ensure that these
operations do not cause data-races.

\mysubsection{ThreadLocal}{th:thread-local}
\emph{ThreadLocal}~\cite{TL-site} is an easy way to achieve thread
confinement~\ref{th:thread-confinement}. It is a wrapper on objects that ensure
that each thread will receive a unique copy of that object.

\begin{lstlisting}[caption={ThreadLocal usage example}, label =
{code:TL-example}]
class ThreadMonkey{
	ThreadLocal<Integer> data = new ThreadLocal<Integer>(){
		@Override
		Integer initialValue(){ return 42};
	}
	
	public void concurrentTaskWithNoInitialization(){
		//....
		useData(data.get());
		//...
	}
	
	public void concurrentTaskWithInitialization(){
		data.set(6*9);
		//...
		useData(data.get());//this call will pass 54 as a parameter.
	}
}
\end{lstlisting}

The \emph{initialValue()} hook is a means of supplying an initial value to any
new thread in the absence of any calls to \emph{set()} before any \emph{get()}
operations. If no initial value is supplied and there is no call to \emph{set()}
withing a new thread then \emph{get()} shall return \textbf{null}. A sequential
program executed using ThreadLocal objects behaves the same as one that doesn't.

\mysection{Profilers}{th:profilers}
Profilers are tools that help us analyze our programs dynamically (at runtime).
This way they can measure memory usage, the frequency and duration of method
calls and trace precisely which program paths are executed for given input.
(information not available with static analysis).

\mysubsection{YourKit Java Profiler 11}{th:yourkit}
YourKit~\cite{YourKit-site} is the profiler we have used to analyze open-source
projects in order to find computationally intensive loops.

\mysection{Static Analysis}{th:static-analysis}
Static analysis is the analysis of computer programs without actually executing
them. It is done on source code and/or object code. It is often employed by
automated tools.

\mysubsection{WALA T.J. Watson Libraries for analysis}{th:wala}
WALA~\cite{wala-site} is an open-source library that provides Java byte-code
analysis.

\mysubsection{Dominator (graph theory)}{th:dominators}
In control flow graphs a node \textbf{d} dominates a node \textbf{n} if every
path from a \textit{start node} to to \textbf{n} must go through \textbf{d}.
WALA gives us a means for calculating dominators based on the implementation
presented in~\cite{dominator-algo}.

\mysubsection{CGNode}{th:cgnode}
During static analysis, methods and/or functions are associated with an abstract
representation called a \emph{Call Graph Node}. \emph{CGNode} is WALA's answer
to these nodes. Also, each \emph{CGNode} contains the control-flow graph
of the abstracted function.

\mysubsection{Call graph}{th: call graph}
The call graph abstracts the call structure of the program by representing the
calls between methods as edges between call graph nodes (in WALA, CGNode).

\mysubsection{Control-flow graph}{th:cfg}
Is a representation in graph form of the path that a program might take during
execution time. The nodes are basic block (uninterrupted lines of code; e.g.
assignments, operations).

\mysubsection{Interprocedural, Finite, Distributive, subset problems}{th:ifds}
Are a special case of graph reachability problems where a finite set of facts is
propagated through the nodes of the graph and the merging of facts accumulated
at path intersections are done through set union or set
intersection~\cite{IFDS}.The algorithm for solving IFDS problems has a very big
advantage over other static analysis algorithms because it offers infinite
context sensitivity, the return edges of method calls are linked to each
individual call site so that when you do step-wise execution on the graph you
return to the unique node from which the method was invoked.