\graphicspath{{content/accesstrace/figures/}}

\mychapter{Access Trace}{chap:access-trace}
\epigraph{\textit{We wants it, we needs it. Must have the precious.}}%
{---\textsc{Gollum}} 

The idea behind this algorithm is the following: after we have determined which
data-races are \slcds we want to take each privatizable field and find its
origin while tracing the path (classes that have owned the object referenced by
the aforementioned field at some point) through which it got into the parallel
context.
Knowing the origin is important to determine initial values, if you recall
listing~\ref{code:jmol:SticksRenderer-TL-naive} you can see that some
\emph{ThreadLocal} fields have an \emph{initialValue()}. When we cannot
determine initial values we fall back to the solution presented in
chapter~\ref{chap:privatization}, which uses the trace information we have
gathered during this phase.

The algorithm uses information from both the call graph(see~\ref{th:cfg}) and
the heap-graph (see~\cite{wala-heapgraph}) to do the above mentioned task.
Basically, it gets an \emph{InstanceKey} associated with a field then goes
through the def-use chain of said field to the previous assignment and gets a
new \emph{InstanceKey}. From this point it iterates through the two stages mentioned
above until it exhausts all the fields that are candidates for privatization. We
treat the data from the access trace as if each individual trace-path is merged
with the union operator.

A very abstract representation of how the access trace looks like can be seen
in Figure~\ref{fig:accesstrace}.

\begin{figure}[h]
	\begin{center}
		\includegraphics[width=10cm]{accesstrace}
		\caption{Abstract representation of an access trace}
		\label{fig:accesstrace}
	\end{center}
\end{figure}

\mysection{Evaluation}{ssec:access-trace-evaluation}
The access trace algorithm has been tested on Jmol and it currently outputs
correct traces for only 30 of the 100 \slcds(see
subsection~\ref{ssec:lcd-evaluation}) detected by our tool.
This inadvertently limits our ability to compute the optimal way of privatizing
objects, this is also the only hindrance in our attempt to build a fully
automated tool.

\mysection{A question of correctness}{sec:access-trace-correctness}
Even though we've solved the problem of propagating the state of the privatized
data from the main thread to the worker threads that perform the parallel
operation with our \emph{ThreadPrivatqe}~\ref{ssec:threadprivate} we have no way
of making the state changes in the worker threads visible again to the outside.

In consequence we have to make sure that the semantics of the program do not
require state change of the privatized data to be propagated outside the
parallel context. This raises the question of how. How do we do this?
