\graphicspath{{content/analysis/figures/}}

\mychapter{Open-source project analysis}{chap:analysis}
\epigraph{\textit{I have approximate answers and possible beliefs in different
degrees of certainty about different things, but I'm not absolutely sure of
anything\ldots}}%
{---\textsc{Richard Feynman}} 

Data-races can manifest themselves in multiple ways so we have used a best
practice when it comes to automated refactoring tools: identifying the common
changes that programmers make in the process of parallelizing an application.
Thus, we started out by analyzing a total of 34 open-source Java projects (a
complete list can be found in appendix~\ref{ap:projects}). We have chosen three
of them as candidates for parallelization: \textbf{Jmol, Art of Illusion,
VASSAL}~\cite{Jmol-site, VASSAL-site, AOI-site}. The reason we chose so few is
because the other ones did not contain any computationally intensive loops in
order to warrant parallelization.

\mysection{The manual process of parallelization}{sec:manual-parallelization}

We started out by running the program with the Yourkit Java Profiler
(see~\ref{th:yourkit}) in order to find the computationally intensive parts of
the program. Since our aim is to build an automated refactoring tool we had to
find code that did not require very high level changes to its structure.
In this thesis I will be using only Jmol~\cite{Jmol-site} as an example. The
most computationally intensive loop of the program is the following:

\begin{lstlisting}[caption={Computationally intensive loop within the
SticksRenderer class}, label = {code:jmol:render_seq}]
private void render_seq(Bond[] bonds){
	for (int i = modelSet.getBondCount(); --i>= 0;) {
		bond = bonds[i];
		if ((bond.getShapeVisibilityFlags() & myVisibilityFlag) != 0)
			renderBond();
    }
}
\end{lstlisting}	

The code of the \emph{renderBond()} method in listing~\ref{code:jmol:render_seq}
is included in annex~\ref{ap:SticksRenderer} because of its sheer size and
complexity. It spans over 6 classes, countless other methods and references a
total of 101 fields 997 times. These numbers make it pretty clear that it is
difficult for a programmer to fully grasp the code in order to ensure its
correctness.
And the fact it is executed in parallel only makes things worse.
Some fields are only read, which automatically makes their usage safe. If they
weren't read only we would have to do a more thorough inspection of each
referencing. It is fairly obvious that this is a tedious, time-consuming and
monotonous task for a human. Moreover, it is extremely error-prone, one missed
reference during this investigation can lead to a data-race, basically
introducing a bug in the program that is known to be very hard to detect and
fix.

A naive, in the sense that no measures were taken to ensure correctness, version
of the parallelized loop can be seen in listing~\ref{code:jmol:render_par}. Line
8 causes a data race; the correct version is presented later in this chapter.

\begin{lstlisting}[caption={The incorrect parallelized version of
listing~\ref{code:jmol:render_seq}}, label = {code:jmol:render_par}]
 private void render_par(Bond[] bonds) {
    ParallelArray<Bond> array = ParallelArray.createUsingHandoff(bonds,
    ParallelArray.defaultExecutor()); array.apply(new Procedure<Bond>() {

      @Override
      public void op(Bond b) {
        //all the necessary data for the computation is provided by this object
        bond = b;
        if ((bond.getShapeVisibilityFlags() & myVisibilityFlag) != 0)
          renderBond();
      }
    });
  }
\end{lstlisting}

\emph{SticksRenderer}, the class that contains the parallel operation has the
field structure presented in Section \ref{code:jmol:SticksRenderer-fields}. All
these fields would be written at some point in the execution of the
\emph{renderBond()} and read later on the execution path causing data-races.

\begin{lstlisting}[caption={Some of the fields in SticksRenderer}, label =
{code:jmol:SticksRenderer-fields}]
public class SticksRenderer extends ShapeRenderer {
  private Atom atomA, atomB;
  private Bond bond;
  private int xA, yA, zA;
  private int xB, yB, zB;
  private int dx, dy;
  private short colixA, colixB;
  private boolean isAntialiased;
  private boolean slabbing;
  private int[] dashDots;
  
  private final Vector3f x = new Vector3f();
  private final Vector3f y = new Vector3f();
  private final Vector3f z = new Vector3f();
  /*everything else was omitted for the sake of brevity*/
}
\end{lstlisting}

We noticed that the fields of \emph{SticksRenderer} from
listing~\ref{code:jmol:SticksRenderer-fields} are being initialized in the
parallel context and used only within the same context. This gave us the idea
that the only thing we had to do to eliminate data races is make sure that each
thread would receive its own unique objects. This can be easily achieved by
using Java's \emph{ThreadLocal}~\ref{th:thread-local}. A very naive method of using
\emph{ThreadLocal} is presented in~\ref{code:jmol:SticksRenderer-TL-naive}.

\begin{lstlisting}[caption={Naive method of privatizing the data}, label =
{code:jmol:SticksRenderer-TL-naive}]
public class SticksRenderer extends ShapeRenderer {
	private ThreadLocal<Atom> atomA = new ThreadLocal<Atom>(), 
							  atomB = new ThreadLocal<Atom>();
	private ThreadLocal<Bond> bond = new ThreadLocal<Bond>();
	private ThreadLocal<Integer> xA = new ThreadLocal<Integer>();
	private ThreadLocal<Integer> yA = new ThreadLocal<Integer>();
	private ThreadLocal<Integer> zA = new ThreadLocal<Integer>();
	private ThreadLocal<Integer> xB = new ThreadLocal<Integer>(), 
								 yB = new ThreadLocal<Integer>(),
								 zB = new ThreadLocal<Integer>();
	private ThreadLocal<Integer> dx = new ThreadLocal<Integer>(),
								 dy = new ThreadLocal<Integer>();
	private ThreadLocal<Short> colixA = new ThreadLocal<Short>(),
							   colixB = new ThreadLocal<Short>();
	private ThreadLocal<Integer> width = new ThreadLocal<Integer>();
	private final ThreadLocal<Vector3f> x = new ThreadLocal<Vector3f>() {
		@Override
		public Vector3f initialValue() {
			return new Vector3f();
		}
	};
	private final ThreadLocal<Vector3f> y = new ThreadLocal<Vector3f>() {
		@Override
		public Vector3f initialValue() {
			return new Vector3f();
		}
	};
	private final ThreadLocal<Vector3f> z = new ThreadLocal<Vector3f>() {
		@Override
		public Vector3f initialValue() {
			return new Vector3f();
		}
	};
\end{lstlisting}

This method is a inefficient because referencing \emph{ThreadLocal} objects
often during the execution causes severe overhead. We performed this
transformation on each individual field that was involved in a data-race and the
difference in performance is from an average sequential running time of 1.4s to
2.2s; so an overhead of roughly 63\%. Thus came the necessity of a better
solution, which not only could reduce the overhead introduced but also minimize
the changes to the code. By converting fields to \emph{ThreadLocal} the
programmer has to update all the references.

The above mentioned drawbacks are eliminated if we encapsulate at a higher level
(i.e. try to privatize the objects that aggregate most fields), which will be
discussed in chapter~\ref{chap:privatization}. Also, an alternative
privatization method is presented in
listing~\ref{code:jmol:SticksRenderer-TL-DTO}. Instead of fetching the data
transfer object for every field reference, we simply fetch it once in every
method and store the reference in a local variable; as seen
in~\ref{code:jmol:DTO-inderection}. This approach causes no measurable overhead.

\begin{lstlisting}[caption={Creating a data transfer object}, label =
{code:jmol:SticksRenderer-TL-DTO}]
public class SticksRenderer extends ShapeRenderer {
	public static class SticksRendererData{
		private Atom atomA, atomB;
  		private Bond bond;
  		private int xA, yA, zA;
  		private int xB, yB, zB;
  		private int dx, dy;
  		private short colixA, colixB;
		private boolean isAntialiased;
  		private boolean slabbing;
  
  		private final Vector3f x = new Vector3f();
  		private final Vector3f y = new Vector3f();
  		private final Vector3f z = new Vector3f();
	}
	
	ThreadLocal<SticksRendererData> data = new ThreadLocal<SticksRendererData>{
		@Override
		public SticksRendererData initialValue() {
			return new SticksRendererData();
		}
	};
}
\end{lstlisting}

\begin{lstlisting}[caption={Referencing a data transfer object}, label =
{code:jmol:DTO-inderection}]
private void renderBond(Bond bond) {
	SticksRendererData localData = this.data.get();
	localData.atomA = bond.getAtom1();
	localData.atomB = bond.getAtom2();
	
	//the rest of the method was avoided for the sake of brevity
}
}
\end{lstlisting}
Using the privatization method presented in
listing~\ref{code:jmol:SticksRenderer-TL-DTO} we can write a correct solution
for the parallelization presented in~\ref{code:jmol:render_par}. As you can see
in~\ref{code:jmol:render_par_correct} on line 6 the field \emph{bond} is
accessed through the encapsulated privatized data object.

\begin{lstlisting}[caption={The correct parallelized version of
listing~\ref{code:jmol:render_seq}}, label = {code:jmol:render_par_correct}]
 private void render_par(Bond[] bonds) {
    ParallelArray<Bond> array = ParallelArray.createUsingHandoff(bonds,
    ParallelArray.defaultExecutor()); array.apply(new Procedure<Bond>() {
      @Override
      public void op(Bond b) {
        data.get().bond = b;
        if ((bond.getShapeVisibilityFlags() & myVisibilityFlag) != 0)
          renderBond();
      }
    });
  }
\end{lstlisting}

Run on an 2.2GHz Intel Core i3 with 4 cores the improvement in performance is
2.3x (from an average of 1.4s to an average of 0.6s).